\subsubsection{Tract}
\label{unit:Tract}

A large rectangular region mapped to the sky with a single map
projection, associated with a particular \unitref{SkyMap}.

\textbf{Dependencies:} SkyMap

\textbf{Value Fields:}
\begin{itemize}
  \item \textbf{tract (int):}
      Unique (with SkyMap) integer identifier for the Tract.
\end{itemize}

\textbf{Table:} \hyperref[tbl:Tract]{Tract}
\begin{table}[!htb]
  {\footnotesize
    \begin{tabular}{| l | l | l | p{0.5\textwidth} |}
      \hline
      \textbf{Name} & \textbf{Type} & \textbf{Attributes} & \textbf{Description} \\
      \hline
      skymap & str & PRIMARY KEY &
              The \unitref{SkyMap} with which this \unitref{Tract} is
              associated.
          \\
      \hline
      tract & int & PRIMARY KEY &
              Unique (with \unitref{SkyMap}) integer identifier for the
              \unitref{Tract}.
          \\
      \hline
      ra & float &  &
              Right ascension of the center of the tract (degrees).
          \\
      \hline
      dec & float &  &
              Declination of the center of the tract (degrees).
          \\
      \hline
      region & bytes &  &
              A spatial region on the sky that bounds the area associated
              with the \unitref{Tract}.  This is expected to be more precise
              than the SkyPixels associated with the \unitref{Visit} (see
              \tblref{TractSkyPixJoin}), but may still be larger than the
              \unitref{Tract} as long as it fully covers it. Regions are
              lsst.sphgeom.ConvexPolygon objects persisted as portable (but
              not human-readable) bytestrings using the encode and decode
              methods.
          \\
      \hline
    \end{tabular}
  }
  \caption{Tract Columns}
  \label{tbl:Tract}
\end{table}
