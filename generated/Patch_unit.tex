\subsubsection{Patch}
\label{unit:Patch}

A rectangular region within a \unitref{Tract}.

\begin{itemize}
  \item Tract
  \item SkyMap
\end{itemize}

\textbf{Value Fields:}
\begin{itemize}
  \item \textbf{patch (int):}
      Unique (with SkyMap and Tract) integer identifier for the Patch.
\end{itemize}

\textbf{Table:} \hyperref[tbl:Patch]{Patch}
\begin{table}[!htb]
  {\footnotesize
    \begin{tabular}{| l | l | l | p{0.5\textwidth} |}
      \hline
      \textbf{Name} & \textbf{Type} & \textbf{Attributes} & \textbf{Description} \\
      \hline
      skymap & str & PRIMARY KEY &
              The \unitref{SkyMap} with which this \unitref{Patch} is
              associated.
          \\
      \hline
      tract & int & PRIMARY KEY &
              The \unitref{Tract} with which this \unitref{Patch} is
              associated.
          \\
      \hline
      patch & int & PRIMARY KEY &
              Unique (with \unitref{SkyMap} and \unitref{Tract}) integer
              identifier for the \unitref{Patch}.
          \\
      \hline
      cell\_x & int & NOT NULL &
              Which column this \unitref{Patch} occupies in the
              \unitref{Tract}'s grid of Patches.
          \\
      \hline
      cell\_y & int & NOT NULL &
              Which row this \unitref{Patch} occupies in the
              \unitref{Tract}'s grid of Patches.
          \\
      \hline
      region & bytes &  &
              A spatial region on the sky that bounds the area associated
              with the \unitref{Patch}.  This is expected to be more precise
              than the SkyPixels associated with the \unitref{Visit} (see
              \tblref{PatchSkyPixJoin}), but may still be larger than the
              \unitref{Patch} as long as it fully covers it. Regions are
              lsst.sphgeom.ConvexPolygon objects persisted as portable (but
              not human-readable) bytestrings using the encode and decode
              methods.
          \\
      \hline
    \end{tabular}
  }
  \caption{Patch Columns}
  \label{tbl:Patch}
\end{table}
