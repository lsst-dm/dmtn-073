\subsubsection{VisitPatchJoin}
\label{join:VisitPatchJoin}

A spatial join table that relates \unitref{Visit} to \unitref{Patch}
via \unitref{SkyPix}. Should be implemented as a view; it may be
materialized as long as it can be kept up to date when new Visits or
SkyMaps are added. If a database UDF is available to determine whether
two regions overlap, we could include that in this view to refine the
results. For now, we will assume that such a UDF is not available.

\textbf{View:} \hyperref[tbl:VisitPatchJoin]{VisitPatchJoin}, defined as:
\begin{verbatim}
  SELECT DISTINCT
  VisitSkyPixJoin.camera,
  VisitSkyPixJoin.visit,
  PatchSkyPixJoin.skymap,
  PatchSkyPixJoin.tract,
  PatchSkyPixJoin.patch
FROM
  VisitSkyPixJoin INNER JOIN PatchSkyPixJoin ON (
    VisitSkyPixJoin.skypix = PatchSkyPixJoin.skypix
  );

\end{verbatim}
\begin{table}[!htb]
  {\footnotesize
    \begin{tabular}{| l | l | l | p{0.5\textwidth} |}
      \hline
      \textbf{Name} & \textbf{Type} & \textbf{Attributes} & \textbf{Description} \\
      \hline
      camera & str & NOT NULL &
              Name of the \unitref{Camera} associated with the
              \unitref{Visit}.
          \\
      \hline
      visit & int & NOT NULL &
              \unitref{Visit} ID
          \\
      \hline
      skymap & str & NOT NULL &
              Name of the \unitref{SkyMap} associated with the
              \unitref{Patch}.
          \\
      \hline
      tract & int & NOT NULL &
              \unitref{Tract} ID
          \\
      \hline
      patch & int & NOT NULL &
              \unitref{Patch} ID
          \\
      \hline
    \end{tabular}
  }
  \caption{VisitPatchJoin Columns}
  \label{tbl:VisitPatchJoin}
\end{table}
