\subsubsection{PhysicalFilter}
\label{unit:PhysicalFilter}

A filter associated with a particular \unitref{Camera}.
PhysicalFilters are used to identify datasets that can only be
associated with a single observation.

\textbf{Dependencies:} Camera

\textbf{Value Fields:}
\begin{itemize}
  \item \textbf{physical\_filter (str):}
      String name of the filter, typically a multi-letter code in a
      convention defined by the Camera (e.g. "HSC-I" or "F775W").
\end{itemize}

\textbf{Table:} \hyperref[tbl:PhysicalFilter]{PhysicalFilter}
\begin{table}[!htb]
  {\footnotesize
    \begin{tabular}{| l | l | l | p{0.5\textwidth} |}
      \hline
      \textbf{Name} & \textbf{Type} & \textbf{Attributes} & \textbf{Description} \\
      \hline
      camera & str & PRIMARY KEY &
              Name of the \unitref{Camera} with which this filter is
              associated.
          \\
      \hline
      physical\_filter & str & PRIMARY KEY &
              String name of the filter, typically a multi-letter code in a
              convention defined by the \unitref{Camera} (e.g. "HSC-I" or
              ``F775W'').
          \\
      \hline
      abstract\_filter & str &  &
              Name of the \unitref{AbstractFilter} with which this filter is
              associated.
          \\
      \hline
    \end{tabular}
  }
  \caption{PhysicalFilter Columns}
  \label{tbl:PhysicalFilter}
\end{table}
