\subsubsection{Visit}
\label{unit:Visit}

A sequence of observations processed together, comprised of one or
more Exposures from the same \unitref{Camera} with the same pointing
and \unitref{PhysicalFilter}. The \unitref{Visit} table contains
metadata that is both meaningful only for science Exposures and the
same for all Exposures in a \unitref{Visit}.

\textbf{Dependencies:} Camera

\textbf{Value Fields:}
\begin{itemize}
  \item \textbf{visit (int):}
      Unique (with camera) integer identifier for this Visit.
\end{itemize}

\textbf{Table:} \hyperref[tbl:Visit]{Visit}
\begin{table}[!htb]
  {\footnotesize
    \begin{tabular}{| l | l | l | p{0.5\textwidth} |}
      \hline
      \textbf{Name} & \textbf{Type} & \textbf{Attributes} & \textbf{Description} \\
      \hline
      camera & str & PRIMARY KEY &
              The \unitref{Camera} used to observe the Exposures associated
              with this \unitref{Visit}.
          \\
      \hline
      visit & int & PRIMARY KEY &
              Unique (with camera) integer identifier for this
              \unitref{Visit}.
          \\
      \hline
      physical\_filter & str & NOT NULL &
              The bandpass filter used for all exposures in this
              \unitref{Visit}.
          \\
      \hline
      datetime\_begin & datetime &  &
              Timestamp of the beginning of the \unitref{Visit}.  This
              should be the same as the datetime\_begin of the first
              \unitref{Exposure} associated with this \unitref{Visit}.
          \\
      \hline
      datetime\_end & datetime &  &
              Timestamp of the end of the \unitref{Visit}.  This should be
              the same as the datetime\_end of the last \unitref{Exposure}
              associated with this \unitref{Visit}.
          \\
      \hline
      exposure\_time & float &  &
              The total exposure time of the \unitref{Visit} in seconds.
              This should be equal to the sum of the exposure\_time values
              for all constituent Exposures (i.e. it should not include time
              between Exposures).
          \\
      \hline
      earth\_rotation\_angle & float &  &
              Earth rotation angle in degrees.
          \\
      \hline
      boresight\_ra & float &  &
              Position of the boresight in right ascension (decimal
              degrees).
          \\
      \hline
      boresight\_dec & float &  &
              Position of the boresight in declination (decimal degrees).
          \\
      \hline
      boresight\_alt & float &  &
              Position of the boresight in altitude (decimal degrees).
          \\
      \hline
      boresight\_az & float &  &
              Position of the boresight in azimuth (decimal degrees).
          \\
      \hline
      boresight\_hour\_angle & float &  &
              Hour angle at the boresight.
          \\
      \hline
      boresight\_parallactic\_angle & float &  &
              Equal to the angle between the North celestial pole and Zenith
              at the boresight. Or, the angular separation between two great
              circle arcs that meet at the object, one passing through the
              North celestial pole, and the other through zenith. For an
              object on the meridian the angle is zero if it is South of
              zenith and pi if it is North of zenith The angle is positive
              for objects East of the meridian, and negative for objects to
              the West.
          \\
      \hline
      boresight\_airmass & float &  &
              Airmass at the boresight, relative to zenith at sea level.
          \\
      \hline
      rot\_angle & float &  &
              Rotation angle of the focal plane w.r.t. nominal.
          \\
      \hline
      local\_era & float &  &
              Local sideral time on the meridian (equivalent, but not equal,
              to Local Mean Sidereal Time).
          \\
      \hline
      seeing & float &  &
              Average seeing, measured as the FWHM of the Gaussian with the
              same effective area (arcsec).
          \\
      \hline
      region & bytes &  &
              A spatial region on the sky that bounds the area covered by
              the \unitref{Visit}.  This is expected to be more precise than
              the region covered by the SkyPixels associated with the
              \unitref{Visit}, but may still be larger than the
              \unitref{Visit} as long as it fully covers it.  Must also
              fully cover all regions in the \tblref{VisitSensorRegion}
              entries associated with this \unitref{Visit}. Regions are
              lsst.sphgeom.ConvexPolygon objects persisted as portable (but
              not human-readable) bytestrings using the encode and decode
              methods.
          \\
      \hline
    \end{tabular}
  }
  \caption{Visit Columns}
  \label{tbl:Visit}
\end{table}
