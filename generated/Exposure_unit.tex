\subsubsection{Exposure}
\label{unit:Exposure}

An observation associated with a particular camera.  All direct
observations are identified with an \unitref{Exposure}, but derived
datasets that may be based on more than one \unitref{Exposure} (e.g.
multiple snaps) are typically identified with Visits instead, even for
cameras that don't have multiple Exposures per \unitref{Visit}.  As a
result, Cameras that don't have multiple Exposures per \unitref{Visit}
will typically have \unitref{Visit} entries that are essentially
duplicates of their \unitref{Exposure} entries. The \unitref{Exposure}
table contains metadata entries that are relevant for calibration
Exposures, and does not duplicate entries in \unitref{Visit} that
would be the same for all Exposures within a \unitref{Visit}.

\textbf{Dependencies:} Camera

\textbf{Value Fields:}
\begin{itemize}
  \item \textbf{exposure (int):}
      Unique (with camera) integer identifier for this Exposure.
\end{itemize}

\textbf{Table:} \hyperref[tbl:Exposure]{Exposure}
\begin{table}[!htb]
  {\footnotesize
    \begin{tabular}{| l | l | l | p{0.5\textwidth} |}
      \hline
      \textbf{Name} & \textbf{Type} & \textbf{Attributes} & \textbf{Description} \\
      \hline
      camera & str & PRIMARY KEY &
              The \unitref{Camera} used to observe the \unitref{Exposure}.
          \\
      \hline
      exposure & int & PRIMARY KEY &
              Unique (with camera) integer identifier for this
              \unitref{Exposure}.
          \\
      \hline
      visit & int &  &
              ID of the \unitref{Visit} this \unitref{Exposure} is
              associated with.  Science observations should essentially
              always be associated with a visit, but calibration
              observations may not be.
          \\
      \hline
      physical\_filter & str & NOT NULL &
              The bandpass filter used for all exposures in this
              \unitref{Visit}.
          \\
      \hline
      snap & int &  &
              If visit is not null, the index of this \unitref{Exposure} in
              the \unitref{Visit}, starting from zero.
          \\
      \hline
      datetime\_begin & datetime &  &
              TAI timestamp of the start of the \unitref{Exposure}.
          \\
      \hline
      datetime\_end & datetime &  &
              TAI timestamp of the end of the \unitref{Exposure}.
          \\
      \hline
      exposure\_time & float &  &
              Duration of the \unitref{Exposure} with shutter open
              (seconds).
          \\
      \hline
      dark\_time & float &  &
              Duration of the \unitref{Exposure} with shutter closed
              (seconds).
          \\
      \hline
    \end{tabular}
  }
  \caption{Exposure Columns}
  \label{tbl:Exposure}
\end{table}
