\subsubsection{SkyPix}
\label{unit:SkyPix}

A pixel in a hierarchical decomposition of the sky (e.g. HTM, Q3C, or
HEALPix; we will select and support just one, but which is TBD). Has
no SQL representation; even a definition table is not necessary, given
that the allowable values and the associated spatial regions are best
computed on-the-fly.  \unitref{SkyPix} units are preferred to
\unitref{SkyMap} (i.e. \unitref{Tract}- \unitref{Patch}) units for
Datasets without any overlap regions (e.g. sharded reference
catalogs). There are also considerable advantages to standardizing on
just one level of the standard pixelization: if all \unitref{SkyPix}
values are at a single level, they can be indexed using standard
B-Trees and compared with simple equality comparison.  In contrast,
comparing \unitref{SkyPix} values at different levels requires
pixelization- specific bit-shifting operations and custom indexes,
which are much harder to implement across multiple RDMSs.  As a
result, we will (at least initially) try to define just a single level
for all \unitref{SkyPix} values.  Our preliminary guess is that this
level should have pixels be approximately (within a factor of
\textasciitilde{}4) of the size of a single \unitref{Sensor} on the
sky.

\textbf{Dependencies:} none

\textbf{Value Fields:}
\begin{itemize}
  \item \textbf{skypix (int):}
      Unique id of a pixel in the hierarchical pixelization, using a
      numbering scheme that also encodes the level of the pixel.
\end{itemize}

\textbf{Table:} none
