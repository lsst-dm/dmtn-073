\begin{tabular}{| l | l | l | p{0.5\textwidth} |}
  \hline
  \textbf{Name} & \textbf{Type} & \textbf{Attributes} & \textbf{Description} \\
  \hline
  dataset\_id & int & PRIMARY KEY &
      a unique autoincrement field used the primary key for dataset.
      \\
  \hline
  dataset\_type\_name & str & NOT NULL &
      the name of the \tblref{DatasetType} associated with this dataset;
      a reference to the \tblref{DatasetType} table.
      \\
  \hline
  run\_id & int & NOT NULL &
      the id of the run that produced this dataset, providing access to
      coarse provenance information.
      \\
  \hline
  quantum\_id & int &  &
      the id of the quantum that produced this dataset, providing access
      to fine-grained provenance information. may be null for datasets
      not produced by running a supertask.
      \\
  \hline
  camera & str &  &
      \\
  \hline
  abstract\_filter & str &  &
      string name for the abstract filter, frequently a single
      character.
      \\
  \hline
  physical\_filter & str &  &
      \\
  \hline
  sensor & str &  &
      \\
  \hline
  visit & int &  &
      \\
  \hline
  exposure & int &  &
      \\
  \hline
  valid\_first & int &  &
      first exposure identifier included in the range (inclusive).  may
      be zero to indicate an open interval.
      \\
  \hline
  valid\_last & int &  &
      last exposure identifier included in the range (inclusive).  may
      be max(int) to indicate an open interval.
      \\
  \hline
  skypix & int &  &
      unique id of a pixel in the hierarchical pixelization, using a
      numbering scheme that also encodes the level of the pixel.
      \\
  \hline
  skymap & str &  &
      \\
  \hline
  tract & int &  &
      \\
  \hline
  patch & int &  &
      \\
  \hline
  label & str &  &
      a string value composed only of letters, numbers, and underscores.
      \\
  \hline
\end{tabular}
